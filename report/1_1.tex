\subsection{Like a hot knife through a Butterworth}

We are asked to design a filter meeting the following requirements:
\begin{itemize}
  \item Passband from 0.2 to 0.3
  \item Stopbands from 0 to 0.1 ad 0.4 to 1
  \item Rippels in passband no greater than 2dB
  \item Stopband attenuation min. 100dB
\end{itemize}

While the headline does imply that a butterworth filter could be used, this
does not allow us to utilize that rippels in both the stop and passband are
accepted.

We use the \emph{Matlab} filter toolbox to design our filer:
\begin{lstlisting}[
style=Matlab-editor,
basicstyle=\ttfamily\footnotesize,
numbers=none]
% Construct an FDESIGN object and call its ELLIP method.
h  = fdesign.bandpass(Fstop1, Fpass1, Fpass2, Fstop2, Astop1, Apass, ...
                      Astop2);
Hd = design(h, 'ellip', 'MatchExactly', match);
\end{lstlisting}
This gives us a $12^{th}$ order filter, compared to that a buttersworth filter
would have to be of order $24$ to meet the same specifications.

Pole-Zero plot of the filter can be seen in fig. \ref{1-1-uncut-zplane},
frequency response (and specification boundaries) can be found in fig.
\ref{1-1-uncut-fresponse} and impulse response in fig.
\ref{1-1-uncut-impresponse}

\stdfignoscale{1-1-uncut-zplane}{ Pole-zero map of a
$12^{th}$ order eliptic bandpass filter with $-3dB$ cutoff frequencies at
$0.3004\pi rad/sample$ and $1.997\pi rad/sample$ }{1-1-uncut-zplane}
\stdfignoscale{1-1-uncut-fresponse}{ Frequency response of a
$12^{th}$ order eliptic bandpass filter with $-3dB$ cutoff frequencies at
$0.3004\pi rad/sample$ and $1.997\pi rad/sample$ }{1-1-uncut-fresponse}

\stdfignoscale{1-1-uncut-impresponse}{ Impulse response of a
$12^{th}$ order eliptic bandpass filter with $-3dB$ cutoff frequencies at
$0.3004\pi rad/sample$ and $1.997\pi rad/sample$ }{1-1-uncut-impresponse}

We find that point $i$ in the impulse response where the maximum amplitude is
below a value of $10\%$ of the maximum amplitude of the whole signal so that:
\begin{equation*}
x[0:\inf] * 10\% > x[i:\inf]
\end{equation*}

This is done with the \emph{Matlab} code:
\begin{lstlisting}[
style=Matlab-editor,
basicstyle=\ttfamily\footnotesize,
numbers=none]
%make list of max of remaining response:
maxrest = uncut_ir;
for i = 1:length(maxrest)
    maxrest(i) = max( abs( uncut_ir(i:end) ) );
    if maxrest(i) < maxval*sig
        break; %stop here, i is now the index, where all samples [i:inf[ < max
    end
end
\end{lstlisting}

This gives an ``effective'' length of $134$ samples. (Had we chosen a
buttersworth filter instead this would have been $146$ samples).

We now use a rectangular window to cut the impulseresponse at $100\%$, $75\%$,
$60\%$, $40\%$ and $10\%$ of the original. The frequency response of the filter
with these impulseresponses as kernes are shown in fig. \ref{1-1-cut-fresponse}

\stdfignoscale{1-1-cut-fresponse}{ Frequency response of an FIR approximation of
an IIR filter using a rectangular window, where the impulse response has been
cut at different lengths.
Notice how the ripple in the passband stays almost constant at $2dB$, while the
attenuation in the stopband is drastically reduced compared to the original IIR
filter. }{1-1-cut-fresponse}

It is quite clear that neither of the filters meet the requirements. As the
impulse length is shortend there, surprisingly, is not that big of a difference
in the filter, until somewhere between $40\%$ and $10\%$ of the length.


Using a hann window, however, we can manipulate the filter respons to get much
closer to the requirements - all without changing the length of the filter. See
fig. \ref{1-1-cut-window-fresponse}.
When applying the window it is important to notice, that the impulse
response is already causal and $x[i] = 0$ for $i<0$, so we only have to apply the half of
the window on the right of the magnutude axis.

\stdfignoscale{1-1-cut-window-fresponse}{ Frequency response of an FIR approximation of
an IIR filter using a hann window, where the impulse response has been
cut at different lengths. Notice how the passband
ripple has been smeared out by convolution in the
frequency domain, this has also made the corners of
the filter rounder, resulting in the cutoff
frequencies to move inside the desired passband. On
the other hand stopband attenuation has ben increased
by a huge factor. }{1-1-cut-window-fresponse}
